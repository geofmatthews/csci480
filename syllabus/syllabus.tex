\documentclass[]{article}
\usepackage{hyperref}
\usepackage[margin=1in]{geometry}
\usepackage{parskip}
\begin{document}

\centerline{\large Syllabus, CSCI 480, Computer Graphics, Fall 2014}

\begin{description}

  \item[Instructor:] Geoffrey Matthews, x3797\\
    {\em Office hours:} MTWF 9:00, CF 469\\
    {\em Email:} geoffrey dot matthews at wwu dot edu
    
\item[Websites:]\mbox{}
\begin{itemize}
  \item For class materials: \url{https://github.com/geofmatthews/csci480}
  \item For turning in homework and
    grading: \url{https://wwu.instructure.com/} 
\end{itemize}

\item[Lectures:] MTWF 10:00-10:50, MH 038

\item[Goals:]  This class is an introduction to fundamental algorithms
  of computer graphics, and to programming in OpenGL.  We will 
  implement several  algorithms for the generation of images in a
  general purpose programming language, and then we will
  study how those methods, and others, have been implemented in
  hardware in the OpenGL pipeline.

By the end of this course the student should be able to:
\begin{itemize}
\item Use basic linear algebra in applications to computer
  graphics. 
\item Use homogeneous coordinates to represent points and vectors.
\item Produce and use geometric transformations
  represented as matrices.
\item Understand the spaces and projections used in 3D graphics.
\item Understand the basic difference between physically based
  rendering and direct rendering, and program each kind of renderer.
\item Program graphical applications in 3D using OpenGL.
\item Understand the OpenGL pipeline.
\item Understand vertex and fragment shaders, and program
  them in GLSL, the OpenGL Shader Language.
\end{itemize}
\item[Text:]  Online notes and web
  pages will be used.  I have assembled relevant wikipedia articles
  into a book that covers most of what we need: \\
  \url{https://en.wikipedia.org/wiki/Book:CSCI_480_Computer_Graphics}
  
\item[Homework:] Usually paper\&pencil math type homework, as assigned
  through the quarter.

\item[Programs:] As assigned through the quarter.  Programming
  language for the course is python. 

\item[Exams:]
  One midterm and one final, as on the schedule below.

  Exams are closed book, with the exception that you may
  consult two pieces of paper during the exam.  You may write or print
  whatever you wish on these pages.

\item[Grading:] ~

\begin{tabular}{|c|c|c|c|}\hline
Homework & Programs &  Midterm & Final \\\hline
25\% & 25\% & 20\% & 30\%  \\\hline
\end{tabular}

\item[Letter grades:] ~

A $\geq$ 90\%  $>$ B  $\geq$ 80\% $>$ C $\geq$
  70\% $>$ D  $\geq$ 60\% $>$ F

  Plus and minus is at the discretion of the instructor.

\item[Special projects:] Students may optionally complete one extra
  credit project for a maximum of 10 percentage points of extra
  credit.  Special projects must: (a) be proposed to the instructor,
  in writing, at least three weeks before the last day of classes,
  (b) be approved by the instructor at that time, (c) must include a
  substantial programming component, (d) must include a writeup with
  an introduction outlining the purpose of the project, a section
  discussing the results, conclusions, {\em etc.}, (e) must include
  well annotated source code, a user's guide, and a programmer's guide
  to the software, and (f) must be complete and turned in before the
  last week of class.  There are no other extra credit opportunities
  for this class.

\item[Academic dishonesty:] Academic dishonesty policy and procedure
  is discussed in the University Catalog, Appendix D.  All students
  should read this section of the catalog.  It consists of
  misrepresentation by deception or other fraudulent means.  In
  computer science courses this frequently takes the form of copying
  another's program, either a fellow student's program, or copying one
  from the web.  Due diligence should be exercised in the labs at all
  times, since both copying and letting someone else copy your program
  are equally culpable.  Do not walk away from your computer in the
  lab without logging out or locking the screen.  Do not share files,
  even if it is just to ``show them something.''  Describe it in
  words, or talk to them in person; never share code.

\item[Collaboration:] Collaboration with your fellow students is a
  good way to learn.  Feel free to share ideas, solve problems, and
  discuss your programs with other students.  However, collaboration
  is {\em not} copying.  All code should be original.  Remember {\bf
    the Simpsons Rule:} after discussing homework with another
  student, you must erase the board and destroy all written notes,
  pictures, files, {\em etc.} that you shared.  After that, you must
  watch a rerun of {\em the Simpsons}, or do something else
  unrelated, for half an hour.  Then you can take the knowledge you
  gained from another student and put it to work, since it is now not
  copying, but learning.  You have made it your own.

\item[Schedule:]  rough guide to the weeks ahead:
  \begin{itemize}
\item Week 1: Introduction, python, pygame
\item Week 2: noise, linear algebra
\item Week 3-4: raytracing: intersection tests, Phong reflection
  model, normals, Whitted raytracing, texture coordinates, implicit
  surfaces, parametric surfaces, derivatives, 
\item Week 5:     transforms, direct rendering, line rendering,
  polygon filling, Phong shading
\item Wednesday, October 28:     Midterm
\item Weeks 6-10:       OpenGL
\item Tuesday, Dec 8:  Final exam, 8:00am.
\end{itemize}


\end{description}

\end{document}
