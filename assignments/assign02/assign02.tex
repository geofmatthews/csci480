\documentclass{article}
\usepackage{hyperref}
\usepackage[margin=1in]{geometry}
\title{Assignment \# 2, CSCI 480\\Fall 2015}

\date{Due date: Friday, Oct 30, midnight.}
\begin{document}
\maketitle

\begin{itemize}

\item Finish the ray tracer provided on the website in the folder
  \verb|raytracer_minimal|.  There are several sections of code marked
  with ``{\tt pass}'' so that they will still compile, but do
  nothing.  Finish each of these as illustrated in the lecture notes:
  \begin{description}
  \item[camera:] {\tt lookAt }
  \item[light:] {\tt PointLight }
  \item[material:] {\tt Image, Reflector }
  \item[shape:]{\tt Plane, PlaneIntersection, Cube,
    QuadricOfMyChoice }
  \item[world:]{\tt MyWorld }
  \end{description}

\item You should implement a single quadric of your choice, but not
  the ellipsoid.  You do {\em not} have to implement rotation for this
  quadric, but you should be able to move it to random locations.
      
\item You should implement a {\tt MyWorld} class that illustrates all
  of the features you implemented.

\item Write modular, well-documented code.  

\item Create some fascinating images.  Use your noise function, or some
  images from it, as procedural or image textures.

\item Create a {\tt main.py} file, which, when run, will open a
  pygame window and create your most interesting image.  This will
  save me time grading and trying to figure out the parameters I need
  to call your function, etc.

\item Zip your code and best images together (don't use tar or
  anything else--I will be writing scripts to unzip and then run
  {\tt main.py}) and submit to Canvas by the due date.

\end{itemize}

\end{document}

