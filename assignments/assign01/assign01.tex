\documentclass{article}
\usepackage{hyperref}
\usepackage[margin=1in]{geometry}
\title{Assignment \# 1, CSCI 480\\Fall 2015}

\date{Due date: Wednesday, Oct 14, midnight.}
\begin{document}
\maketitle

\begin{itemize}

\item Implement a value noise function in python as outlined in the lecture
notes and on the ``Perlin Noise'' website which actually discusses
value noise, and not true Perlin noise:
\url{http://freespace.virgin.net/hugo.elias/models/m_perlin.htm}

  \item Use a shuffled array for random numbers rather than a
    pseudo-random number generator like the one on the website.

\item Implement a pink noise function which uses your value noise
  function.


\item Use the pink noise function, together with the framework for
  making images I gave you in {\tt pygamecolors.py}, to make some
  cool pictures.
  
  Try to be creative!  Make marble, dirt, clouds,
  landscapes, something interesting!
  
\item Write modular, well-documented code.  Some of you are new at
  Python, so I don't expect objects/etc.  However, decomposing the
  main algorithm into intelligible pieces is mandatory.

\item Create a {\tt main.py} folder (it can be your entire program, if
  you like), which, when run, will open a pygame window and create
  your most interesting image.  This will save me time grading and
  trying to figure out the parameters I need to call your function,
  etc.

\item Zip your code and best images together (don't use tar or
  anything else--I will be writing scripts to unzip and then run
  {\tt main.py}) and submit to Canvas by the due date.

\end{itemize}

\end{document}

