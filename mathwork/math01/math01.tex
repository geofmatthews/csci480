\documentclass{article}
\usepackage[margin=1in]{geometry}
\usepackage{tikz}
\usetikzlibrary{arrows}

\newcommand{\vect}[1]{\langle #1 \rangle}
\begin{document}
\noindent{\bf CSCI 480, Fall 2015, Math Homework \# 1}
\bigskip

\noindent Name\hrulefill Number \hrulefill

Show your work.  Explain, in each case, how you got a particular
numerical result.  You may, of course, use a computer for numeric
computations, but which expressions you used to get each result must
be made clear.  For example, you can't just put down 12, you must put
down $2^2 + 2^2 + 2^2 = 12$.  You can't just put down 4.443, you
must put down $\pi\cdot\sqrt{2} = 4.443$

If you have trouble writing neat math, consider learning and using \LaTeX.

\begin{enumerate}
\item Let $v=\vect{2,2,1}$ and $w=\vect{1,-2,0}$.  Find the following:
  \begin{enumerate}
  \item $v\cdot w$
\vfill
  \item $v\times w$
\vfill
  \item The vector projection of $w$ on $v$.
\vfill
  \end{enumerate}

\item Orthogonalize each of the following vectors:
  \begin{enumerate}
  \item $v_1 = \vect{\frac{\sqrt{2}}{2}, \frac{\sqrt{2}}{2}, 0}$
\vfill
  \item $v_2 = \vect{-1,1,-1}$
\vfill
  \item $v_3 = \vect{0,-2,-2}$
\vfill
  \end{enumerate}

\item Find the cosine of the angle between the vectors
  $v=\vect{1,2,3}$ and $w=\vect{3,2,1}$
\vfill

\item Let $v = \vect{4,3,-1}$.  Decompose $v$ into the sum of two
    vectors, one of which is parallel to $n=\vect{\frac{\sqrt{2}}{2},
      \frac{\sqrt{2}}{2}, 0}$, and the other is perpendicular.
\begin{enumerate}
\item Parallel:
\vfill
\item Perpendicular:
  \vfill
  \end{enumerate}
    \newpage
  \item 
    For the picture below, find:
    \begin{enumerate}
      \item The coordinates of $q$ in the frame $\langle p, u, v\rangle$
\vfill
      \item The coordinates of $q$ in the frame $\langle p, u, w\rangle$
\vfill
      \item The coordinates of $q$ in the frame $\langle p, v, w\rangle$
\vfill
      \item The coordinates of $p$ in the frame $\langle q, u, v\rangle$
\vfill
      \item The coordinates of $p$ in the frame $\langle q, u, w\rangle$
\vfill
      \item The coordinates of $p$ in the frame $\langle q, v, w\rangle$
\vfill
    \end{enumerate}

    \tikzset{>=latex}
    \begin{tikzpicture}
      \draw[dotted] (0,0) grid (9.9,9.9);
      \draw[thick,<->] (10,0) node[anchor=north west] {$x$ axis}
      -- (0,0) -- (0,10) node[anchor=south east] {$y$ axis};
      \foreach \x in {0,1,2,3,4,5,6,7,8,9}
        \draw (\x cm,2pt) -- (\x cm,-2pt) node[anchor=north] {$\x$};
      \foreach \y in {0,1,2,3,4,5,6,7,8,9}
        \draw (2pt,\y) -- (-2pt,\y cm) node[anchor=east] {$\y$};

      \fill (3,7) circle (3pt) node[anchor=north west] {$p$};
      \fill (9,3) circle (3pt) node[anchor=north west] {$q$};

      \draw[->,thick] (2,1) -- (4,1) node[midway,anchor=south] {$u$};
      \draw[->,thick] (5,3) -- (7,5) node[midway,anchor=south] {$w$};
      \draw[->,thick] (5,9) -- (9,5) node[midway,anchor=south] {$v$};

    \end{tikzpicture}

\end{enumerate}
\end{document}
