% latex foo.tex 
% dvips -Poutline -G0 foo.dvi -o 
% ps2pdf -dPDFSETTINGS#/prepress foo.ps
\documentclass[slidestop,xcolor=pst]{beamer}
\usepackage{fancyvrb}
\usepackage{hyperref}



\usepackage[english]{babel}

\usepackage[latin1]{inputenc}

\newcommand{\sect}[1]{\begin{frame}\frametitle{#1}}

\title[Computer Graphics, CSCI 480]
{Computer Graphics, CSCI 480
}

\subtitle{} % (optional)

\author[Geoffrey Matthews]
{Geoffrey Matthews}
% - Use the \inst{?} command only if the authors have different
%   affiliation.

\institute[WWU/CS]
{
  Department of Computer Science\\
  Western Washington University
}
% - Use the \inst command only if there are several affiliations.
% - Keep it simple, no one is interested in your street address.

\date{Fall 2014}

% If you have a file called "university-logo-filename.xxx", where xxx
% is a graphic format that can be processed by latex or pdflatex,
% resp., then you can add a logo as follows:

%\pgfdeclareimage[height=0.5cm]{university-logo}{WWULogoProColor}
%\logo{\pgfuseimage{university-logo}}

% If you wish to uncover everything in a step-wise fashion, uncomment
% the following command: 

%\beamerdefaultoverlayspecification{<+->}

\begin{document}


\begin{frame}
  \titlepage
\end{frame}

%\begin{frame}
%  \frametitle{Outline}
%  \tableofcontents
%  % You might wish to add the option [pausesections]
%\end{frame}

\sect{Outline}
\begin{itemize}
\item Introduction to fundamental algorithms of computer graphics.
\item Study implementations in general purpose language (python).
\item Study both raytracing and direct rendering.
\item See how important algorithms have been built into hardware and
  accessed by a special purpose library, OpenGL.
\end{itemize}

\end{frame}

\sect{Grading}
\begin{itemize}
\item 25\%: Homework assignments.  Pencil and paper, math work.  
\item 25\%: Programming assignments.  Turn in on Canvas.
\item 20\%: Midterm.  Wednesday, October 28
\item 30\%: Final.  Tuesday, December 8, 8:00am.
\end{itemize}
\end{frame}

\sect{Extra Credit}
\begin{itemize}
\item Up to 10 extra credit points can be earned by doing a project.
\item Project must be proposed, in writing, to the instructor three
  weeks before the last day of class.
\item Project must be approved by the instructor, possibly with
  modifications, within 72 hours.
\item Must include a substantial programming component.
\item Must include a writeup.
\item Must include well annotated source code.
\item Must be complete and turned in before the last day of class.
\end{itemize}
\end{frame}

\sect{Academic Dishonesty}
\begin{itemize}
\item Read Appendix D of the University Catalog.
\item Do not share code.
\item Follow the {\bf Simpson's Rule}
\end{itemize}
\end{frame}

\sect{Python, pygame, numpy, pyopengl}
\begin{itemize}
\item Programming language for the course: {\bf python}
\item {\bf Pygame}:  a python package to facilitate game programming
\item {\bf Numpy}:  a python package for numerical operations, including
  linear algebra
\item {\bf Pyopengl}:  a python package wrapping the OpenGL language
\end{itemize}
\end{frame}

\sect{Why python?}
\begin{itemize}

\item It is easy.
\item It is easy to do hard things; many data structures and
  programming styles out of the box. 
\item It is OS neutral.
\item It has easy access to numerical libraries.
\item It has easier interface to OpenGL.
\item It has pygame, so 3D applications can be given game-like controls.

\end{itemize}

\end{frame}

\sect{Learning python and pygame}
\begin{itemize}
  \item Work through a python tutorial, {\em e.g.}:\\ \url{http://interactivepython.org/courselib/static/thinkcspy/index.html}
\item Work through some of the pygame examples (installed with
  download),
  especially:\\
  \url{https://www.pygame.org/docs/tut/chimp/ChimpLineByLine.html}
  \item Structure of {\tt pygamecolors.py}
  \end{itemize}
\end{frame}

\sect{Schedule}
  \begin{itemize}
\item Week 1: Introduction, python, pygame
\item Week 2: noise, linear algebra
\item Week 3-4: raytracing: intersection tests, Phong reflection
  model, normals, Whitted raytracing, texture coordinates, implicit
  surfaces, parametric surfaces, derivatives, 
\item Week 5:     transforms, direct rendering, line rendering,
  polygon filling, Phong shading
\item Wednesday, October 28:     Midterm
\item Weeks 6-10:       OpenGL
\item Tuesday, Dec 8:  Final exam, 8:00am.
\end{itemize}
%\begin{columns}
%\column{0.5\textwidth}
%\column{0.5\textwidth}
%\end{columns}
\end{frame}


\end{document}
