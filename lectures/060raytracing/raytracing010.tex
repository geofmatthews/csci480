% latex foo.tex 
% dvips -Poutline -G0 foo.dvi -o 
% ps2pdf -dPDFSETTINGS#/prepress foo.ps
\documentclass[slidestop,xcolor=pst]{beamer}
\usepackage{etex}
\usepackage{fancyvrb}
\usepackage{hyperref}
%\usepackage{pstricks,pst-tree,pst-node,pst-plot,pst-3dplot}
\usepackage{graphicx}

\newcommand{\mygraph}[2]{\includegraphics[width=#1\textwidth]{figures/#2}}
\newcommand{\mygraphc}[2]{\centerline{\includegraphics[width=#1\textwidth]{figures/#2}}}

\newcommand{\sect}[1]{
\section{#1}
\begin{frame}[fragile]\frametitle{#1}
}

\newcommand{\bi}{\begin{itemize}}
\newcommand{\ei}{\end{itemize}}

\mode<presentation>
{
%  \usetheme{Madrid}
  % or ...

%  \setbeamercovered{transparent}
  % or whatever (possibly just delete it)
}

\usepackage[english]{babel}

\usepackage[latin1]{inputenc}

\title[Computer Graphics, CSCI 480, Ray Tracing I]
{
Ray Tracing, Part I
}

\subtitle{} % (optional)

\author[Geoffrey Matthews]
{Geoffrey Matthews}
% - Use the \inst{?} command only if the authors have different
%   affiliation.

\institute[WWU/CS]
{
  Department of Computer Science\\
  Western Washington University
}
% - Use the \inst command only if there are several affiliations.
% - Keep it simple, no one is interested in your street address.

\date{Fall 2015}

% If you have a file called "university-logo-filename.xxx", where xxx
% is a graphic format that can be processed by latex or pdflatex,
% resp., then you can add a logo as follows:

%\pgfdeclareimage[height=0.5cm]{university-logo}{WWULogoProColor}
%\logo{\pgfuseimage{university-logo}}

% If you wish to uncover everything in a step-wise fashion, uncomment
% the following command: 

%\beamerdefaultoverlayspecification{<+->}

\newcommand{\myref}[1]{\small\item\url{#1}}
\newcommand{\myreff}[1]{\scriptsize\item\url{#1}}

\begin{document}


\begin{frame}
  \titlepage
\end{frame}

\sect{Readings}
\bi
\item
  \url{https://www.siggraph.org/education/materials/HyperGraph/raytrace/rtrace0.htm}
\item
  \url{https://www.cs.unc.edu/~rademach/xroads-RT/RTarticle.html}
\item \url{http://www.cs.utah.edu/~shirley/books/fcg2/rt.pdf}
\item \url{http://www.povray.org/}
\item \url{http://www.pbrt.org/}
  \ei
  
\end{frame}

\sect{Ray Traced Images Achieve Maximal Realism}
\mygraph{1}{800px-Glasses_800_edit.png}
\end{frame}



%\begin{frame}
%  \frametitle{Outline}
%  \tableofcontents
%  % You might wish to add the option [pausesections]
%\end{frame}
\sect{Two ways of rendering a picture: object order}
\mygraph{}{objectorderrendering.png}

\begin{itemize}
\item For each object in the world, find the colors it would put on the screen.
\end{itemize}
\end{frame}

\sect{Two ways of rendering a picture: image order}
\mygraph{}{imageorderrendering.png}
\begin{itemize}
\item For each pixel on the screen, find the objects that would color it.
\end{itemize}
\end{frame}

\sect{Ray casting, a simplified ray tracing}
\mygraph{}{imagecasting.png}
\begin{itemize}
\item Given an eye position and a pixel position, construct a ray.
\item Project the ray into the scene and find the closest intersection.
\item Use object to compute color.
\end{itemize}
\end{frame}

\sect{Camera}
\mygraph{.5}{camera01.png}
\begin{minipage}[b]{2.25in}
\begin{itemize}
\item A frame: {\bf origin} point, and {\bf right}, {\bf up}, {\bf forward} vectors.
\item A distance, width and height for the image plane.
\end{itemize}
\[ \langle p,r,u,f,d,w,h \rangle \]
\begin{itemize}
\item Would it matter if $d,w,h$ were all doubled?
\end{itemize}
\end{minipage}
\end{frame}

\sect{Finding vectors from the eye to the image plane}
\mygraph{.5}{mapping02.png}
\begin{minipage}[b]{2.25in}
\begin{itemize}
\item Give an expression for the upper left corner.
  \item Give an expression for the upper right corner.
\item Give an expression for a point 30\% of the way across the image
  plane and 10\% up from the bottom.
\end{itemize}
\end{minipage}

\[ \langle p,r,u,f,w,h \rangle \]
\end{frame}


\sect{Other camera representations}
\mygraph{.5}{camera01.png}
\begin{minipage}[b]{2.25in}
\begin{itemize}
\item Lookat:  eye point, lookat point, up vector, width, aspect ratio.
\item Eye and four points
\item Eye, lower left corner, two vectors
\item Eye and four vectors
\end{itemize}
\end{minipage}
\end{frame}


\sect{Normalized screen coordinates}
\mygraph{.35}{mapping01.png}\hfill
\mygraph{.35}{normalsquare.png}
\begin{itemize}
\item {\em Normalized} screen coordinates map the entire surface to
  the $(0,1)\times(0,1)$ square.
\item Suppose the screen is $640\times480$, with origin in the upper
  left, and we have a point at $(300,400)$ on the screen.
\item What are the point's normalized screen coordinates? \pause
  \[
  \left(\frac{300}{640}, 1-\frac{400}{480}\right)
  \]
  \pause
  \item Why did I use $s$ and $t$ and not $w$ and $h$?
\end{itemize}
\end{frame}

\sect{Mapping from Pixel to Camera Ray}
\mygraph{.4}{mapping01.png}
\hfill
\mygraph{.5}{mapping02.png}

\begin{itemize}
\item Given a pixel position on the screen, find the ray in the camera.
\item Map screen position to normalized position in $(0,1)\times(0,1)$
\item Map normalized position in $(0,1)\times(0,1)$ to vector in world space.
\end{itemize}
\end{frame}

\sect{Ray casting process}
\bi
\item Input: a camera and a set of objects
\item Output: an image
\item For each pixel in the image:
  \bi
  \item Find the ray in the camera for that pixel.
  \item For all objects in the set:
    \bi
    \item find the closest in front of the camera that intersects the ray
    \ei
  \item Find the color of that object at the intersection point.
  \item Color the pixel in the image with that color.
  \ei
\ei
\end{frame}

\sect{Intersecting a ray and a sphere}
\mygraph{.9}{intersectingraysphere.png}
\begin{itemize}
  \item Sphere defined by center and radius.
\item Ray defined by point and vector.
\item Assume sphere is centered at origin (replace $p$ with $p-c$).
\item Equation to solve?
\pause
\item Solve for $t$:  $|p+tv|^2 = r^2$
\end{itemize}
\end{frame}

\sect{Solving quadratic}
\begin{eqnarray*}
|p+tv|^2 &=& (p+tv)\cdot(p+tv)\\
&=& \sum_i (p_i + tv_i)(p_i + tv_i)\\
&=& \sum_i \left(p_i^2 + 2p_iv_it + v_i^2t^2\right)\\
&=& \sum_i p_i^2 + 2\sum_ip_iv_it + \sum_iv_i^2t^2\\
&=& (p\cdot p) + 2(p\cdot v) t+ (v\cdot v) t^2
\end{eqnarray*}
So, in the quadratic $at^2 + bt + c = 0$,
\begin{eqnarray*}
a &=& v\cdot v \ \ \ (= 1 \mbox{~if you normalized your rays})\\
b &=& 2p\cdot v\\
c &=& p\cdot p - r^2
\end{eqnarray*}
\end{frame}

\sect{Intersecting a ray and a sphere}
\mygraph{}{intersectingraysphere.png}
\begin{itemize}
  \item What equation would we have to solve if we did this in world
    coordinates? 
\pause
\item $|(p+tv) - c|^2 = r^2$
\item Much simpler in object coordinates (replace $p$ with $p-c$).
\end{itemize}
\end{frame}

\sect{Still something missing ...}
\mygraph{0.5}{threespheresflat.png}
\mygraph{0.5}{threespheres.png}

\end{frame}

\end{document}
