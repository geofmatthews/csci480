% latex foo.tex 
% dvips -Poutline -G0 foo.dvi -o 
% ps2pdf -dPDFSETTINGS#/prepress foo.ps
\documentclass[slidestop,xcolor=pst]{beamer}
\usepackage{etex}
\usepackage{fancyvrb}
\usepackage{hyperref}
%\usepackage{pstricks,pst-tree,pst-node,pst-plot,pst-3dplot}
\usepackage{graphicx}

\newcommand{\mygraph}[2]{\includegraphics[width=#1\textwidth]{figures/#2}}
\newcommand{\mygraphc}[2]{\centerline{\includegraphics[width=#1\textwidth]{figures/#2}}}

\newcommand{\sect}[1]{
\section{#1}
\begin{frame}[fragile]\frametitle{#1}
}

\newcommand{\bi}{\begin{itemize}}
\newcommand{\ei}{\end{itemize}}
\newcommand{\myref}[1]{\small\item\url{#1}}
\newcommand{\myreff}[1]{\scriptsize\item\url{#1}}

\mode<presentation>
{
  \usetheme{Madrid}
  % or ...

%  \setbeamercovered{transparent}
  % or whatever (possibly just delete it)
}

\usepackage[english]{babel}

\usepackage[latin1]{inputenc}

\title[Computer Graphics, CSCI 480, Ray Tracing II]
{
Ray Tracing, Part II
}

\subtitle{} % (optional)

\author[Geoffrey Matthews]
{Geoffrey Matthews}
% - Use the \inst{?} command only if the authors have different
%   affiliation.

\institute[WWU/CS]
{
  Department of Computer Science\\
  Western Washington University
}
% - Use the \inst command only if there are several affiliations.
% - Keep it simple, no one is interested in your street address.

\date{Fall 2015}

% If you have a file called "university-logo-filename.xxx", where xxx
% is a graphic format that can be processed by latex or pdflatex,
% resp., then you can add a logo as follows:

\pgfdeclareimage[height=0.5cm]{university-logo}{WWULogoProColor}
\logo{\pgfuseimage{university-logo}}

% If you wish to uncover everything in a step-wise fashion, uncomment
% the following command: 

%\beamerdefaultoverlayspecification{<+->}

\begin{document}


\begin{frame}
  \titlepage
\end{frame}



\sect{Shading: Phong Reflection}

\mygraphc{1}{Phong_components_version_4.png}



\begin{itemize}
\item Phong reflection model, a combination of three simple shaders
\item A {\em phenomenological} model, not a {\em physical } one.
\end{itemize}


\end{frame}

\sect{Phong reflection}
\mygraphc{1}{Phong_components_version_4.png}



 Colors calculated using three vectors:

\bi
\item Orientation of the surface (the {\em normal})
\item Direction toward {\em light}
\item Direction toward camera (or {\em eye})
\ei

\end{frame}

\sect{Spheres: finding the normal at a point}
\mygraphc{.5}{findingnormalonsphere.png}
\begin{itemize}
\item How do we find the normal?\pause
\item $n = p-c$
  \hfill (may need to normalize)
\end{itemize}

\end{frame}

\sect{Spheres: finding the eye vector and the light vector}
\begin{itemize}
\item How do we find the vector pointing toward the camera?
\pause
\\  {\em camera\_position} - {\em point\_on\_sphere}
\pause
\item How do we find the vector pointing toward the light?
\pause
\item Depends on the kind of light.
  \bi
\item {\em Distant lights}, like the sun, are a fixed direction.
\item {\em Point lights}, are located at a point:
   {\em light\_position} - {\em point\_on\_sphere}
\ei
\end{itemize}
\end{frame}

\sect{Shading: Phong Reflection}

\mygraphc{1}{Phong_components_version_4.png}


\begin{itemize}
\item Given {\bf normal}, {\bf light}, and {\bf eye} vectors, how do
  we compute each of these factors?
\end{itemize}


\end{frame}

\sect{The Ambient Term}
\mygraphc{1}{Phong_components_version_4.png}

\begin{itemize}
\item Light comes reflected and mixed from all objects in the environment.
\item Approximate this with a small amount of white light.
\item Without this we would get totally black shadows.
\item Doesn't use any of the vectors---totally flat.
\end{itemize}
\end{frame}

\sect{The Diffuse Term}
\mygraphc{1}{Phong_components_version_4.png}
\bi
\item Gives the shading falloff of light.
  \item Side oriented toward light:  full object color.
\item Side oriented away from light: black.
\ei
\end{frame}

\sect{The Diffuse Term:  Lambertian Reflection}

\mygraphc{0.5}{diffuseterm02.png}


\begin{itemize}
\item Objects with rough surfaces reflect light equally in all directions.
\item Light energy {\bf coming off} surface in any direction will be
  proportional to the amount of light {\bf falling on} the surface.
\end{itemize}
\end{frame}

\sect{The Diffuse Term:  Lambertian Reflection}

\mygraphc{}{diffuseterm.png}

\begin{itemize}
\item Light energy falling on a surface is proportional to the cosine
  of the angle of incidence of the light source.
\item Therefore the diffuse term will be proportional to the cosine of the
  angle of incidence of the light source.
\item Depends only on the {\bf light} and {\bf normal} vectors.
\end{itemize}

\end{frame}

\sect{The Specular Term}
\mygraphc{1}{Phong_components_version_4.png}
\bi
\item Shiny spots are tiny blurry images of the light source.
\item They depend on {\bf normal}, {\bf light}, and {\bf eye} vectors.
\item If you move your head, the shiny spots will move, but the
  lambertian shade will stay the same.
\ei
\end{frame}

\sect{The Specular Term}
\mygraph{}{specularterm.png}
\begin{itemize}
\item Smooth surfaces act a bit like mirrors.
\item Intensity of light will fall off more or less rapidly from the
  ideal (mirror) reflection vector.
\item How do you calculate the reflection vector?
\end{itemize}
\end{frame}

\sect{The Reflection Vector}
\mygraphc{}{reflectionvector.png}
\begin{itemize}
\item Assume all vectors are normalized, find $r$
\pause 
\item $r = \ell - 2(\ell - (n\cdot \ell)n)$
\pause
\item Use $\cos(\phi) = r\cdot e$
\end{itemize}
\end{frame}

\sect{Shininess}
\mygraphc{}{shininess.png}
\begin{itemize}
\item $\cos(x)^i$ for $i \in \{1,2,4,8,16,32\}$
\end{itemize}
\end{frame}

\sect{Specular reflection}
\mygraphc{.7}{specularballs.png}
\begin{itemize}
\item Specular coefficient in $(0.25, 0.5, 0.75)$
\item Shininess in $(3,9,200)$
\end{itemize}

\end{frame}

\sect{The Halfway Vector: a speed hack}
\mygraphc{.75}{halfwayvector.png}
\begin{itemize}
\item Use vector halfway between $e$ and $\ell$
\item If $n = h$, we get the brightest possible reflection.
\item Use $\cos(\beta) = h\cdot n$ for the falloff.
\item This angle $\beta$ is about half the angle $\phi$ found before.
\item We can adjust the shininess to handle that.
\end{itemize}
\end{frame}

\sect{The Halfway Vector}
\mygraphc{}{Blinn_phong_comparison.png}
\end{frame}




\sect{Add up all the terms}
\[
k_a C_a + \sum_{m \in \mbox{lights}} (k_d(L_m\cdot N)C_{m,d}
                + k_s(R_m\cdot E)^\alpha C_{m,s})
\]
\begin{eqnarray*}
C_a,C_s,C_d &=& \mbox{ambient, specular, diffuse colors}\\
k_a,k_s,k_d &=& \mbox{ambient, specular, diffuse reflection constants}\\
\alpha &=& \mbox{shininess}\\
L_m &=& \mbox{light vector}\\
N &=& \mbox{surface normal}\\
E &=& \mbox{eye vector}\\
R_m &=& \mbox{light vector reflected about normal}\\
  &=& 2(L_m\cdot N)N - L_m
\end{eqnarray*}
\end{frame}


\sect{Phong reflection}
\mygraphc{}{Phong_components_version_4.png}
\[
k_a C_a + \sum_{m \in \mbox{lights}} (k_d(L_m\cdot N)C_{m,d}
                + k_s(R_m\cdot E)^\alpha C_{m,s})
\]
\end{frame}


\sect{Phong examples}
\mygraphc{0.8}{phongexamples.png}
\end{frame}


\sect{Microfacets}

\mygraphc{.5}{microfacets.png}
\begin{itemize}
\item Assume a surface is made up of tiny mirrors.
\item The statistical distribution of these facets will determine the
  reflection in each direction.
\item More sophisticated stochastic models
  such as these give better approximations to some
  surfaces than Phong reflection.
\end{itemize}

\end{frame}

\end{document}

\end{document}
