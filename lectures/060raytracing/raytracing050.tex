% latex foo.tex 
% dvips -Poutline -G0 foo.dvi -o 
% ps2pdf -dPDFSETTINGS#/prepress foo.ps
\documentclass[slidestop,xcolor=pst]{beamer}
\usepackage{etex}
\usepackage{fancyvrb}
\usepackage{hyperref}
%\usepackage{pstricks,pst-tree,pst-node,pst-plot,pst-3dplot}
\usepackage{graphicx}

\newcommand{\mygraph}[2]{\includegraphics[width=#1\textwidth]{figures/#2}}
\newcommand{\mygraphc}[2]{\centerline{\includegraphics[width=#1\textwidth]{figures/#2}}}

\newcommand{\sect}[1]{
\section{#1}
\begin{frame}[fragile]\frametitle{#1}
}

\mode<presentation>
{
  \usetheme{Madrid}
  % or ...

%  \setbeamercovered{transparent}
  % or whatever (possibly just delete it)
}

\usepackage[english]{babel}

\usepackage[latin1]{inputenc}

\title[Computer Graphics, CSCI 480, Ray Tracing IV]
{
Ray Tracing, Part IV
}

\subtitle{} % (optional)

\author[Geoffrey Matthews]
{Geoffrey Matthews}
% - Use the \inst{?} command only if the authors have different
%   affiliation.

\institute[WWU/CS]
{
  Department of Computer Science\\
  Western Washington University
}
% - Use the \inst command only if there are several affiliations.
% - Keep it simple, no one is interested in your street address.

\date{Fall 2015}

% If you have a file called "university-logo-filename.xxx", where xxx
% is a graphic format that can be processed by latex or pdflatex,
% resp., then you can add a logo as follows:

\pgfdeclareimage[height=0.5cm]{university-logo}{WWULogoProColor}
\logo{\pgfuseimage{university-logo}}

% If you wish to uncover everything in a step-wise fashion, uncomment
% the following command: 

%\beamerdefaultoverlayspecification{<+->}

\newcommand{\bi}{\begin{itemize}}
\newcommand{\ei}{\end{itemize}}
\newcommand{\myref}[1]{\small\item\url{#1}}
\newcommand{\myreff}[1]{\scriptsize\item\url{#1}}

\begin{document}


\begin{frame}
  \titlepage
\end{frame}

\sect{A simple scene}
\mygraphc{.8}{simplescene.png}
\begin{itemize}
\pause
\item How do we get more than one color on an object?
\end{itemize}
\end{frame}


\sect{Textures and UV mapping}
\mygraphc{}{UVMapping.png}
\begin{itemize}
\item Need to map point on surface to point in image.
\end{itemize}
\end{frame}

\sect{2D and 3D mapping}
\mygraphc{.8}{UV_mapping_checkered_sphere.png}
\begin{itemize}
\item Can be {\em procedural} rather than image texture.
\end{itemize}
\end{frame}

\sect{A simple scene}
\mygraphc{0.8}{simplescene.png}
\begin{itemize}
\pause\item How did we get the shadow?
\end{itemize}
\end{frame}

\sect{Shadows}
\mygraphc{0.7}{shadowscene.png}
\begin{itemize}
\item Cast a ray from intersection to lights.
\item Do not need closest intersection, can quit after one.
\pause\item Colors in shadows?
\end{itemize}
\end{frame}

\sect{False Self-intersections}
\mygraphc{0.8}{speckles.png}
\begin{itemize}
\item Numeric problems with intersections.
\end{itemize}
\end{frame}

\sect{Noise texture}
\mygraphc{0.85}{noisescene.png}
\end{frame}

\sect{Alternate scene}
\mygraphc{0.85}{alternatescene.png}
\end{frame}

\sect{Noise texture bumpmapped}
\mygraphc{0.85}{bumpnoise.png}
\end{frame}

\sect{Aliasing}
\mygraphc{.7}{aliasingplot.png}

\begin{itemize}
\item Sample the black line at widely spaced gaps.

\item Smoothly connecting the samples gives a signal of a much longer
  wavelength. 
\item The high frequency signal is masquerading, or {\bf aliased} as a
  low frequency signal.
\item All computer graphics is done at {\bf pixels}, which are
  regularly spaced {\bf samples}.  {\em Pixels are not little squares!}
\item Aliasing is a constantly recurring problem in computer graphics.

\end{itemize}

\end{frame}

\sect{Aliasing in the simple scene}
\mygraphc{.5}{simplescene.png}

\begin{itemize}
\item Some visible artifacts
\begin{itemize}
\item Jaggies on the edge of the sphere.
\item Large patches of color on the ground.
\item New patterns in the distant clouds.
\end{itemize}

\item {Some solutions to aliasing}
\begin{itemize}
\item Sample at random points in pixel area.
\item Resample multiple points in the pixel area.
\end{itemize}
\end{itemize}
\end{frame}

\sect{Reflections:  Whitted Raytracing}
\mygraphc{0.8}{reflectionscene.png}
\end{frame}

\sect{Reflections: Recursion Limit}
\mygraphc{0.8}{reflection00.png}
\end{frame}

\sect{Reflections: Recursion Limit}
\mygraphc{0.8}{reflection01.png}
\end{frame}

\sect{Reflections: Recursion Limit}
\mygraphc{0.8}{reflection02.png}
\end{frame}

\sect{Reflections: Recursion Limit}
\mygraphc{0.8}{reflection03.png}
\end{frame}

\sect{Reflections: Recursion Limit}
\mygraphc{0.8}{reflection04.png}
\end{frame}

\sect{Reflections: Recursion Limit}
\mygraphc{0.8}{reflection05.png}
\end{frame}

\sect{Reflections: Recursion Limit}
\mygraphc{0.8}{reflection06.png}
\end{frame}


\sect{Refractions:  Whitted Raytracing}
\mygraphc{0.8}{refraction12.png}
\begin{itemize}
\pause \item What about shadows if the object is transparent?  Caustics?
\end{itemize}
\end{frame}

\sect{The Ray Tree}
\mygraphc{}{raytree.png}
\end{frame}

\sect{Speeding up ray tracing}
\begin{itemize}
\item Embarassingly parallel
\item Object partitioning
\end{itemize}
\end{frame}


\sect{\bf Readings}
\begin{itemize}
\myref{http://scratchapixel.com/lessons/3d-basic-lessons/lesson-1-writing-a-simple-raytracer/}
\myref{http://en.wikipedia.org/wiki/Ray_tracing_(graphics)}
\myref{http://www.cs.unc.edu/~rademach/xroads-RT/RTarticle.html}
\myref{http://en.wikipedia.org/wiki/Phong_lighting}
\myreff{http://www.wiziq.com/tutorial/162719-6-837-7-Ray-Tracing-Computer-Graphics}
\myreff{http://ocw.mit.edu/courses/electrical-engineering-and-computer-science/}
\end{itemize}
{\bf Software:}
\begin{itemize}
\myref{http://pbrt.org/}
\myref{http://www.luxrender.net/en_GB/index}
\myref{http://www.povray.org/}
\myref{http://www.yafaray.org/}
\myref{http://radsite.lbl.gov/radiance/HOME.html}
\myref{http://www-graphics.stanford.edu/~cek/rayshade/rayshade.html}
\end{itemize}

\end{frame}

\end{document}
