% latex foo.tex 
% dvips -Poutline -G0 foo.dvi -o 
% ps2pdf -dPDFSETTINGS#/prepress foo.ps
\documentclass{beamer}
\usepackage{fancyvrb}
\usepackage{hyperref}

\usepackage{graphicx}


\newcommand{\sect}[1]{
\section{#1}
\begin{frame}[fragile]\frametitle{#1}
}


\mode<presentation>
{
%  \usetheme{Madrid}
  % or ...

%  \setbeamercovered{transparent}
  % or whatever (possibly just delete it)
}

\usepackage[english]{babel}

\usepackage[latin1]{inputenc}

\title
{
Raster Operations
}

\subtitle{} % (optional)

\author[Geoffrey Matthews]
{Geoffrey Matthews}
% - Use the \inst{?} command only if the authors have different
%   affiliation.

\institute[WWU/CS]
{
  Department of Computer Science\\
  Western Washington University
}
% - Use the \inst command only if there are several affiliations.
% - Keep it simple, no one is interested in your street address.

\date{Fall 2012}

% If you have a file called "university-logo-filename.xxx", where xxx
% is a graphic format that can be processed by latex or pdflatex,
% resp., then you can add a logo as follows:

%\pgfdeclareimage[height=0.5cm]{university-logo}{WWULogoProColor}
%\logo{\pgfuseimage{university-logo}}

% If you wish to uncover everything in a step-wise fashion, uncomment
% the following command: 

%\beamerdefaultoverlayspecification{<+->}

\begin{document}


\begin{frame}
  \titlepage
\end{frame}


\newcommand{\myref}[1]{\small\item\url{#1}}
\newcommand{\myreft}[1]{\footnotesize\item\url{#1}}

%\begin{frame}
%  \frametitle{Outline}
%  \tableofcontents
%  % You might wish to add the option [pausesections]
%\end{frame}


\sect{Reading}

\begin{itemize}
\myref{http://en.wikipedia.org/wiki/Bresenham's_line_algorithm}
\myref{http://www.cs.helsinki.fi/group/goa/mallinnus/lines/bresenh.html}
\myref{https://en.wikipedia.org/wiki/Scanline_rendering}
\end{itemize}

\end{frame}

\sect{Filling in a triangle}
\begin{itemize}
\item Keep a table of {\bf min} and {\bf max} values for each row of the 
projection.
\item As each point of the line is calculated by the line algorithm,
update min and max for that row.
\item After the lines have been drawn, fill in between the min and max.
\end{itemize}
\end{frame}

\sect{Interpolating other quantities}

Note that as we step $x$ from its initial value to its final value, we
also step $y$ from its initial value to its final value.  We could
also step some other value, such as
\begin{itemize}
\item  $z$
\item a greyscale or color value
\item a normal vector
\item texture coordinates
\end{itemize}
or anything else from an initial value to a final value, thus
LERPing all values from one endpoint to the other.

\end{frame}


\end{document}
