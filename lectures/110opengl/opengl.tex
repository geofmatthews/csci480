% latex foo.tex 
% dvips -Poutline -G0 foo.dvi -o 
% ps2pdf -dPDFSETTINGS#/prepress foo.ps
\documentclass{beamer}
\usepackage{fancyvrb}
\usepackage{hyperref}

\usepackage{graphicx}


\newcommand{\sect}[1]{
\section{#1}
\begin{frame}[fragile]\frametitle{#1}
}


\mode<presentation>
{
  \usetheme{Madrid}
  % or ...

%  \setbeamercovered{transparent}
  % or whatever (possibly just delete it)
}

\usepackage[english]{babel}

\usepackage[latin1]{inputenc}

\title
{
Beginning OpenGL
}

\subtitle{} % (optional)

\author[Geoffrey Matthews]
{Geoffrey Matthews}
% - Use the \inst{?} command only if the authors have different
%   affiliation.

\institute[WWU/CS]
{
  Department of Computer Science\\
  Western Washington University
}
% - Use the \inst command only if there are several affiliations.
% - Keep it simple, no one is interested in your street address.

\date{}

% If you have a file called "university-logo-filename.xxx", where xxx
% is a graphic format that can be processed by latex or pdflatex,
% resp., then you can add a logo as follows:

%\pgfdeclareimage[height=0.5cm]{university-logo}{WWULogoProColor}
%\logo{\pgfuseimage{university-logo}}

% If you wish to uncover everything in a step-wise fashion, uncomment
% the following command: 

%\beamerdefaultoverlayspecification{<+->}

\begin{document}


\begin{frame}
  \titlepage
\end{frame}


\newcommand{\myref}[1]{\small\item\url{#1}}
\newcommand{\myreft}[1]{\footnotesize\item\url{#1}}

%\begin{frame}
%  \frametitle{Outline}
%  \tableofcontents
%  % You might wish to add the option [pausesections]
%\end{frame}


\sect{Reading}


\centerline{\fbox{\begin{minipage}{3in}
{\bf Warning!}  Many (most?) tutorials on the web are obsolete.
Make sure anything you read deals with OpenGL 3.X or later.
\end{minipage}}}

\begin{itemize}

\myref{http://openglbook.com/}
\myref{http://www.opengl-tutorial.org/}
\myref{http://www.arcsynthesis.org/gltut/}
\myref{http://duriansoftware.com/joe/An-intro-to-modern-OpenGL.-Table-of-Contents.html}
\myref{https://open.gl/}
\myref{http://ogldev.atspace.co.uk/index.html}
\end{itemize}

\end{frame}



\end{document}
